\documentclass[addpoints]{exam} % addpoints برای نمایش نمره


\RequirePackage[dvipsnames]{xcolor}
\RequirePackage{tcolorbox}
\RequirePackage{listings}
\usepackage{xepersian}
\usepackage{geometry}
\usepackage{graphicx}
\usepackage{enumitem}
\usepackage[utf8]{inputenc}
\usepackage{amsmath, amssymb}



\renewcommand{\labelitemi}{\scalebox{0.7}{$\bullet$}}

\geometry{
  a4paper,
  left=25mm,
  right=25mm,
  top=25mm,
  bottom=25mm
}

\newtcolorbox{mathbox}[1][]{
  colback=gray!5,
  left=1mm,
  right=1mm,
  leftrule=0pt,
  rightrule=1pt,
  toprule=0pt,
  bottomrule=0pt,
  sharp corners
}


\lstset{
    framerule=1pt,
    frame=tb,
    emphstyle={\small\ttfamily\bfseries\color{Orange}},
    numbers=left,
    numberstyle= \tiny\color{black},
    basicstyle = \small\ttfamily,
    keywordstyle    = \bfseries\color{BrickRed},
    identifierstyle = \bfseries\color{black},
    stringstyle     = \bfseries\color{ForestGreen},
    commentstyle    = \bfseries\color{Violet},
    breaklines      =   true,
    columns         =   fixed,
    basewidth       =   .5em,
    backgroundcolor=\color{Gray!5},
    tabsize=2,
    showspaces=false,
    showstringspaces=false,
}

\settextfont[Path=font/,Extension=.ttf,BoldFont=B Zar Bold]{B Zar}
\setlatintextfont[Scale=0.833]{Times New Roman}



\runningfooter{}{\thepage}{}
\firstpagefooter{}{\thepage}{}

\begin{document}


\begin{titlepage}
    \newcommand{\HRule}{\rule{\linewidth}{0.5mm}}

\newgeometry{top=1cm, bottom=2cm, left=1cm, right=1cm} 

\begin{flushright}
\begin{minipage}{0.15\textwidth}
\includegraphics[scale=.1]{figure/logo.png}
\end{minipage}
\begin{minipage}{0.5\textwidth}
\begin{flushright}
\textbf{\large انجمن علمی مهندسی کامپیوتر}\\
\large دانشگاه مازندران
\end{flushright}
\end{minipage}
\end{flushright}



\begin{center}

\vspace*{8\baselineskip}

{\LARGE 
%------------------------------------------------------------------------------------
  به نام خدا
%------------------------------------------------------------------------------------
}

\HRule \\[0.4cm]
\textbf{\Huge
%------------------------------------------------------------------------------------
    پاسخ تمرین اول ساختمان داده‌ها و الگوریتم
%------------------------------------------------------------------------------------
}\\
\vspace*{0.5\baselineskip} 
{\large
%------------------------------------------------------------------------------------
    الگوریتم‌ها و حل مسئله - پیچیدگی زمانی - الگوریتم های بازگشتی
%------------------------------------------------------------------------------------
}

\HRule \\[1.5cm]

\vfill

\begin{minipage}{0.4\textwidth}
\begin{center} \Large
%------------------------------------------------------------------------------------
    استاد:
%------------------------------------------------------------------------------------
\end{center}
\begin{center} \large
%------------------------------------------------------------------------------------
    دکتر روستایی
%------------------------------------------------------------------------------------
\end{center}
\end{minipage}

\vspace*{2\baselineskip} 

\begin{minipage}{0.4\textwidth}
\begin{center} \Large
%------------------------------------------------------------------------------------
    استادیار‌:
%------------------------------------------------------------------------------------
\end{center}
\begin{center} \large
%------------------------------------------------------------------------------------
عرشیا عموزاد\\
%------------------------------------------------------------------------------------
\end{center}
\end{minipage}

\vspace*{12\baselineskip} 

\today

\end{center}


\end{titlepage}

\begin{questions}

\question
یادآوری: اگر \(f(n)\) و \(g(n)\) توابعی صعودی باشند و \(\log(f(n))\) از \(\log(g(n))\) رشد بیشتری داشته باشد، آنگاه \(f(n)\) از \(g(n)\) رشد بیشتری دارد.\\
البته در صورتی که رشد لگاریتم توابع از مرتبه‌ی یکسانی شود، نمی‌توان درباره هم مرتبه بودن توابع اظهار نظر کرد.\\
\begin{flushleft}
    \(\log(g(n)) = \log(n^{\log n}) = \log^2 n\) \\ \vspace*{0.5\baselineskip}
    \(\log(f(n)) = \log(\log^n n) = n\log(\log(n))\)\\ \vspace*{0.5\baselineskip}
    \(\lim_{n \to \infty} \frac{\log(f(n))}{\log(g(n))} = \lim_{n \to \infty} \frac{n\log(\log(n))}{\log^2(n)}\)\\ \vspace*{0.5\baselineskip}
    \(\log n = m, n = 2^{m}\)\\ \vspace*{0.5\baselineskip} 
    \(\lim_{n \to \infty} \frac{2^{m}\log(m)}{m^2} = \infty \Rightarrow g(n) = O(f(n))\)
\end{flushleft}

\question
مانند سوال قبل رفتار می کنیم.\\
\begin{flushleft}
    
\(\log{f(n)} = \log{n}.\log{4} = 2\log n = \Theta(\log n)\)\\
\(\log{g(n)} = \log^2(n) = \Theta(\log^2 n)\)\\
\(\log{h(n)} = 2\log(\log(n)) = \Theta(\log(\log(n)))\) \\
\(\Rightarrow h(n) \leq f(n) \leq g(n)\)

\end{flushleft}


\question
عمل جستجوی خطی اگر \lr{target} در \lr{arr[1]} باشد. بهترین حالت است و تعداد مقایسه ها \(O(1)\) است. اگر \lr{target} در \lr{arr[n]} باشد،
یا اصلا نباشد، بدترین حالت است و \lr{n} مقایسه انجام می‌شود.\\
بررسی حالت متوسط معمولا دشوار است زیرا باید با در نظر گرفتن تمام حالات ممکن میانگین گرفت.\\
اگر \lr{target} با احتمال \lr{p} در \lr{arr} باشد در نتیجه با احتمال \lr{(1-p)} در آن وجود ندارد. در این صورت داریم:\\
\begin{flushleft}
    \(\frac{p}{n}(\sum_{i=1}^{n} i) + (1-p)n = \frac{p(n+1)}{2} + (1-p)n\)
\end{flushleft}
که با \(p=0.5\) می‌شود: \(\frac{3n+1}{4}\)



\question
خیر فرض کنید \(f(n) = n\) و \(g(n) = n^{1+\sin{n}}\). این مثال درستی نتایج ذکر شده را نقض می‌کند.

\question
یادآوری: \\
\begin{flushleft}
    \(a^b = e^{b.\ln{a}}\)\\
\end{flushleft}
طبق نکته‌ی بالا خواهیم داشت:\\
\begin{flushleft}
    \(f_3(n) = f_5(n) = e^{\log{n}.\log{\log{n}}}\)
\end{flushleft}
حال توابع را بصورت لگاریتمی با هم مقایسه می‌کنیم:
\begin{flushleft}
    \(\ln f_1 \sim 1.01 \ln{n}\)\\
    \(\ln f_2 \sim (\sqrt{n}).\ln{2}\)\\
    \(\ln f_4 \sim (ln{n})^2\)\\
    \(\ln f_3 = \ln f_5 \sim \ln{n}.\ln{\ln{n}}\)\\

\end{flushleft}

با مقایسه لگاریتم توابع، ترتیب صعودی توابع از کوچک به بزرگ خواهد شد:\\
\begin{flushleft}
    \(f_1 < f_3 \equiv f_5 < f_4 < f_2\)
\end{flushleft}



\end{questions}

\end{document}