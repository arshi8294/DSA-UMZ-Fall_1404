\documentclass[addpoints]{exam} % addpoints برای نمایش نمره


\RequirePackage[dvipsnames]{xcolor}
\RequirePackage{tcolorbox}
\RequirePackage{listings}
\usepackage{hyperref}
\usepackage{xepersian}
\usepackage{geometry}
\usepackage{graphicx}
\usepackage{enumitem}
\usepackage[utf8]{inputenc}
\usepackage{amsmath, amssymb}



\renewcommand{\labelitemi}{\scalebox{0.7}{$\bullet$}}

\geometry{
  a4paper,
  left=25mm,
  right=25mm,
  top=25mm,
  bottom=25mm
}

\newtcolorbox{mathbox}[1][]{
  colback=gray!5,
  left=1mm,
  right=1mm,
  leftrule=0pt,
  rightrule=1pt,
  toprule=0pt,
  bottomrule=0pt,
  sharp corners
}


\lstset{
    framerule=1pt,
    frame=tb,
    emphstyle={\small\ttfamily\bfseries\color{Orange}},
    numbers=left,
    numberstyle= \tiny\color{black},
    basicstyle = \small\ttfamily,
    keywordstyle    = \bfseries\color{BrickRed},
    identifierstyle = \bfseries\color{black},
    stringstyle     = \bfseries\color{ForestGreen},
    commentstyle    = \bfseries\color{Violet},
    breaklines      =   true,
    columns         =   fixed,
    basewidth       =   .5em,
    backgroundcolor=\color{Gray!5},
    tabsize=2,
    showspaces=false,
    showstringspaces=false,
}

\settextfont[Scale=1.1,Path=../../font/,Extension=.ttf,BoldFont=B Zar Bold]{B-NAZANIN}
\setlatintextfont[Scale=0.900]{Times New Roman}



\runningfooter{}{\thepage}{}
\firstpagefooter{}{\thepage}{}

\begin{document}


\begin{titlepage}
    \newcommand{\HRule}{\rule{\linewidth}{0.5mm}}

\newgeometry{top=1cm, bottom=2cm, left=1cm, right=1cm} 

\begin{flushright}
\begin{minipage}{0.15\textwidth}
\includegraphics[scale=.1]{../../figure/logo.png}
\end{minipage}
\begin{minipage}{0.5\textwidth}
\begin{flushright}
\textbf{\large انجمن علمی مهندسی کامپیوتر}\\
\large دانشگاه مازندران
\end{flushright}
\end{minipage}
\end{flushright}



\begin{center}

\vspace*{8\baselineskip}

{\LARGE 
%------------------------------------------------------------------------------------
  به نام خدا
%------------------------------------------------------------------------------------
}

\HRule \\[0.4cm]
\textbf{\Huge
%------------------------------------------------------------------------------------
    تمرین عملی دوم ساختمان داده‌ها و الگوریتم
%------------------------------------------------------------------------------------
}\\
\vspace*{0.5\baselineskip} 
{\large
%------------------------------------------------------------------------------------
درخت و کاربردهایش
%------------------------------------------------------------------------------------
}

\HRule \\[1.5cm]

\vfill

\begin{minipage}{0.4\textwidth}
\begin{center} \Large
%------------------------------------------------------------------------------------
    استاد:
%------------------------------------------------------------------------------------
\end{center}
\begin{center} \large
%------------------------------------------------------------------------------------
    دکتر روستایی
%------------------------------------------------------------------------------------
\end{center}
\end{minipage}

\vspace*{2\baselineskip} 

\begin{minipage}{0.4\textwidth}
\begin{center} \Large
%------------------------------------------------------------------------------------
    استادیار‌:
%------------------------------------------------------------------------------------
\end{center}
\begin{center} \large
%------------------------------------------------------------------------------------
عرشیا عموزاد\\
%------------------------------------------------------------------------------------
\end{center}
\end{minipage}

\vspace*{12\baselineskip} 

\today

\end{center}


\end{titlepage}





\newpage
\thispagestyle{empty} 

\begin{center}
{\Huge \textbf{نحوه‌ی ارسال و ارائه‌ی پاسخ تمرین}}
\end{center}

\vspace*{2\baselineskip}

\begin{flushright}
{\Large \textbf{مهلت و نوع فایل های ارسالی:}}
\end{flushright}

\begin{flushright}
\begin{itemize}[rightmargin=1cm, leftmargin=3cm]
    \item پاسخنامه باید شامل فایل های اجرایی \lr{(.py)} و فایل \lr{PDF} شرح کدها باشد.
    \item مهلت ارسال: \textbf{1404/09/30} تا ساعت \textbf{23:59}
    \item  پاسخ‌های خود را به آیدی تلگرام \lr{Arshi82\_94} بفرستید.
    \item روز ارائه‌ی تمرین عملی بعد از هماهنگی در گروه تلگرامی درس اعلام خواهد شد.
\end{itemize}
\end{flushright}


\vspace*{\baselineskip}

\begin{flushright}
{\Large \textbf{ملاحظات پیاده‌سازی:}}
\end{flushright}

\begin{flushright}
\begin{itemize}[rightmargin=1cm, leftmargin=3cm]
    \item از پیاده‌سازی آماده کتابخانه‌ها \textbf{جلوگیری} شود.
    \item در پیاده‌سازی های خود از داده ساختار های آموخته شده(لیست پیوندی، صف، پشته و...) استفاده کنید.
    \item  فایل های \lr{template} کلاس ها در لینک گیت هاب زیر موجود است.(به نوع خروجی های توابع دقت شود)

\end{itemize}
\end{flushright}


\vspace*{15\baselineskip}

\begin{center}
{\large \textbf{موفق باشید}}
\end{center}



\newpage

\begin{questions}
\question
\subsection*{مسئله 1: مجموع اعداد باینری مسیرهای ریشه به برگ}
یک درخت دودویی داده شده که هر گره مقدار 0 یا 1 دارد. هر مسیر از ریشه به برگ نشان‌دهنده یک عدد باینری است (با ارزش‌ترین رقم مربوط به ریشه است). باید مجموع اعداد باینری تمام مسیرهای ریشه به برگ را محاسبه کنید.

\begin{flushright}
تابع \lr{\textbf{sumRootToLeaf}} را پیاده‌سازی کنید:
\end{flushright}

\begin{enumerate}
    \item \lr{\textbf{sumRootToLeaf(root)}}: 
    \\ورودی: ریشه درخت دودویی (از نوع \lr{TreeNode} یا \lr{None})
    \\خروجی: عدد صحیح (مجموع اعداد باینری همه مسیرهای ریشه به برگ)
\end{enumerate}

\vspace*{0.5\baselineskip}
\textbf{فرمت ورودی:}\\
ورودی به صورت یک شیء از کلاس \lr{TreeNode} به شما داده می‌شود. ساختار کلاس \lr{TreeNode} به صورت زیر است:

\begin{latin}
\begin{lstlisting}[language=Python]
class TreeNode:
    def __init__(self, val=0, left=None, right=None):
        self.val = val      # مقدار گره (0 یا 1)
        self.left = left    # زیردرخت چپ
        self.right = right  # زیردرخت راست
\end{lstlisting}
\end{latin}

\vspace*{0.5\baselineskip}
\textbf{مثال ۱:}\\

\begin{latin}
\begin{lstlisting}[language=Python]
#        1
#       / \
#      0   1
#     / \ / \
#    0  1 0  1

root = TreeNode(
    1,
    TreeNode(0, TreeNode(0), TreeNode(1)),
    TreeNode(1, TreeNode(0), TreeNode(1))
)

# خروجی تابع باید باشد:
sumRootToLeaf(root)  # ==> 22
\end{lstlisting}
\end{latin}

\vspace*{0.5\baselineskip}
\textbf{توضیح مثال ۱:}\\
چهار مسیر از ریشه به برگ وجود دارد:
\begin{itemize}[leftmargin=2cm, rightmargin=2cm]
    \item مسیر ۱: \lr{1 → 0 → 0} = عدد باینری \lr{100} = ۴ (در مبنای ۱۰)
    \item مسیر ۲: \lr{1 → 0 → 1} = عدد باینری \lr{101} = ۵
    \item مسیر ۳: \lr{1 → 1 → 0} = عدد باینری \lr{110} = ۶
    \item مسیر ۴: \lr{1 → 1 → 1} = عدد باینری \lr{111} = ۷
\end{itemize}
مجموع = ۴ + ۵ + ۶ + ۷ = ۲۲

\vspace*{0.5\baselineskip}
\textbf{مثال ۲:}\\

\begin{latin}
\begin{lstlisting}[language=Python]
root = TreeNode(0)

sumRootToLeaf(root)  # ==> 0
\end{lstlisting}
\end{latin}

\vspace*{0.5\baselineskip}
\textbf{توضیح مثال ۲:}\\
یک مسیر داریم: \lr{0} = عدد باینری \lr{0} = ۰

\vspace*{0.5\baselineskip}
\textbf{مثال ۳:}\\

\begin{latin}
\begin{lstlisting}[language=Python]
#        1
#       / \
#      1   0

root = TreeNode(1, TreeNode(1), TreeNode(0))

sumRootToLeaf(root)  # ==> 5
\end{lstlisting}
\end{latin}

\vspace*{0.5\baselineskip}
\textbf{توضیح مثال ۳:}\\
دو مسیر از ریشه به برگ وجود دارد:
\begin{itemize}[leftmargin=2cm, rightmargin=2cm]
    \item مسیر ۱: \lr{1 → 1} = عدد باینری \lr{11} = ۳
    \item مسیر ۲: \lr{1 → 0} = عدد باینری \lr{10} = ۲
\end{itemize}
مجموع = ۳ + ۲ = ۵

\vspace*{0.5\baselineskip}
\textbf{نکات قابل توجه:}
\begin{itemize}[rightmargin=1cm, leftmargin=3cm]
    \item تعداد گره‌های درخت بین ۱ تا ۱۰۰۰ است.
    \item مقدار هر گره ۰ یا ۱ است.
    \item پاسخ تضمین شده است که در محدوده عدد ۳۲ بیتی علامت‌دار قرار می‌گیرد.
    \item پشته\lr{ (Stack)}** برای پیمایش درخت می‌توانید استفاده کنید.
    \item در صورت مواجهه با درخت خالی (\lr{None})، مقدار ۰ را برگردانید.
\end{itemize}


\vspace*{2\baselineskip}


\question
\subsection*{مسئله 2: یافتن بزرگ‌ترین مقدار در هر سطر درخت}
یک درخت دودویی داده شده است. باید بزرگ‌ترین مقدار موجود در هر سطر (سطح) درخت را پیدا کرده و به صورت یک لیست بازگردانید. سطرها از ریشه (سطر ۰) شروع می‌شوند.

\begin{flushright}
تابع \lr{\textbf{largestValues}} را پیاده‌سازی کنید:
\end{flushright}

\begin{enumerate}
    \item \lr{\textbf{largestValues(root)}}: 
    \\ورودی: ریشه درخت دودویی (از نوع \lr{TreeNode} یا \lr{None})
    \\خروجی: لیستی از اعداد صحیح (بزرگ‌ترین مقدار هر سطر)
\end{enumerate}

\vspace*{0.5\baselineskip}
\textbf{فرمت ورودی:}\\
ورودی به صورت یک شیء از کلاس \lr{TreeNode} به شما داده می‌شود. ساختار کلاس \lr{TreeNode} به صورت زیر است:

\begin{latin}
\begin{lstlisting}[language=Python]
class TreeNode:
    def __init__(self, val=0, left=None, right=None):
        self.val = val      # مقدار گره
        self.left = left    # زیردرخت چپ
        self.right = right  # زیردرخت راست
\end{lstlisting}
\end{latin}

\vspace*{0.5\baselineskip}
\textbf{مثال ۱:}\\

\begin{latin}
\begin{lstlisting}[language=Python]
#        1
#       / \
#      3   2
#     / \   \
#    5   3   9

root = TreeNode(1,
                TreeNode(3, TreeNode(5), TreeNode(3)),
                TreeNode(2, None, TreeNode(9)))

largestValues(root)  # ==> [1, 3, 9]
\end{lstlisting}
\end{latin}

\vspace*{0.5\baselineskip}
\textbf{توضیح مثال ۱:}\\
\begin{itemize}[leftmargin=2cm, rightmargin=2cm]
    \item سطر ۰ (ریشه): \lr{[1]} \(\leftarrow\) بزرگترین: ۱
    \item سطر ۱: \lr{[3, 2]} \(\leftarrow\) بزرگترین: ۳
    \item سطر ۲: \lr{[5, 3, 9]} \(\leftarrow\) بزرگترین: ۹
\end{itemize}

\vspace*{0.5\baselineskip}
\textbf{مثال ۲:}\\

\begin{latin}
\begin{lstlisting}[language=Python]
#        1
#       / \
#      2   3

root = TreeNode(1, TreeNode(2), TreeNode(3))

largestValues(root)  # ==> [1, 3]
\end{lstlisting}
\end{latin}

\vspace*{0.5\baselineskip}
\textbf{توضیح مثال ۲:}\\
\begin{itemize}[leftmargin=2cm, rightmargin=2cm]
    \item سطر ۰: \lr{[1]} \(\leftarrow\) بزرگترین: ۱
    \item سطر ۱: \lr{[2, 3]} \(\leftarrow\) بزرگترین: ۳
\end{itemize}

\vspace*{0.5\baselineskip}
\textbf{مثال ۳:}\\

\begin{latin}
\begin{lstlisting}[language=Python]
#        7

root = TreeNode(7)

largestValues(root)  # ==> [7]
\end{lstlisting}
\end{latin}

\vspace*{0.5\baselineskip}
\textbf{مثال ۴:}\\

\begin{latin}
\begin{lstlisting}[language=Python]
root = None

largestValues(root)  # ==> []
\end{lstlisting}
\end{latin}

\vspace*{0.5\baselineskip}
\textbf{مثال ۵:}\\

\begin{latin}
\begin{lstlisting}[language=Python]
#        -1
#        / \
#      -2  -3
#      /   / \
#    -5  -4  -6

root = TreeNode(-1,
                TreeNode(-2, TreeNode(-5)),
                TreeNode(-3, TreeNode(-4), TreeNode(-6)))

largestValues(root)  # ==> [-1, -2, -4]
\end{lstlisting}
\end{latin}

\vspace*{0.5\baselineskip}
\textbf{توضیح مثال ۵:}\\
\begin{itemize}[leftmargin=2cm, rightmargin=2cm]
    \item سطر ۰: \lr{[-1]} \(\leftarrow\) بزرگترین: ۱-
    \item سطر ۱: \lr{[-2, -3]} \(\leftarrow\) بزرگترین: ۲-
    \item سطر ۲: \lr{[-5, -4, -6]} \(\leftarrow\) بزرگترین: ۴-
\end{itemize}

\vspace*{0.5\baselineskip}
\textbf{نکات قابل توجه:}
\begin{itemize}[rightmargin=1cm, leftmargin=3cm]
    \item تعداد گره‌های درخت در محدوده \([0, 10^4]\) است.
    \item می‌توانید از صف برای نگهداری گره‌های هر سطح استفاده نمایید.
    \item در صورت مواجهه با درخت خالی (\lr{None})، لیست خالی برگردانید.
    % \item پیچیدگی زمانی مطلوب: \(O(n)\) که \(n\) تعداد گره‌هاست.
\end{itemize}

\end{questions}

\end{document}