
\documentclass[addpoints]{exam} % addpoints برای نمایش نمره


\RequirePackage[dvipsnames]{xcolor}
\RequirePackage{tcolorbox}
\RequirePackage{listings}
\usepackage{xepersian}
\usepackage{geometry}
\usepackage{graphicx}
\usepackage{enumitem}
\usepackage[utf8]{inputenc}
\usepackage{amsmath, amssymb}



\renewcommand{\labelitemi}{\scalebox{0.7}{$\bullet$}}

\geometry{
  a4paper,
  left=25mm,
  right=25mm,
  top=25mm,
  bottom=25mm
}

\newtcolorbox{mathbox}[1][]{
  colback=gray!5,
  left=1mm,
  right=1mm,
  leftrule=0pt,
  rightrule=1pt,
  toprule=0pt,
  bottomrule=0pt,
  sharp corners
}


\lstset{
    framerule=1pt,
    frame=tb,
    emphstyle={\small\ttfamily\bfseries\color{Orange}},
    numbers=left,
    numberstyle= \tiny\color{black},
    basicstyle = \small\ttfamily,
    keywordstyle    = \bfseries\color{BrickRed},
    identifierstyle = \bfseries\color{black},
    stringstyle     = \bfseries\color{ForestGreen},
    commentstyle    = \bfseries\color{Violet},
    breaklines      =   true,
    columns         =   fixed,
    basewidth       =   .5em,
    backgroundcolor=\color{Gray!5},
    tabsize=2,
    showspaces=false,
    showstringspaces=false,
}

\settextfont[Scale=1.1,Path=font/,Extension=.ttf,BoldFont=B Zar Bold]{B-NAZANIN}
\setlatintextfont[Scale=0.900]{Times New Roman}



\runningfooter{}{\thepage}{}
\firstpagefooter{}{\thepage}{}

\begin{document}


\begin{titlepage}
    \newcommand{\HRule}{\rule{\linewidth}{0.5mm}}

\newgeometry{top=1cm, bottom=2cm, left=1cm, right=1cm} 

\begin{flushright}
\begin{minipage}{0.15\textwidth}
\includegraphics[scale=.1]{figure/logo.png}
\end{minipage}
\begin{minipage}{0.5\textwidth}
\begin{flushright}
\textbf{\large انجمن علمی مهندسی کامپیوتر}\\
\large دانشگاه مازندران
\end{flushright}
\end{minipage}
\end{flushright}



\begin{center}

\vspace*{8\baselineskip}

{\LARGE 
%------------------------------------------------------------------------------------
  به نام خدا
%------------------------------------------------------------------------------------
}

\HRule \\[0.4cm]
\textbf{\Huge
%------------------------------------------------------------------------------------
    تمرین دوم ساختمان داده‌ها و الگوریتم
%------------------------------------------------------------------------------------
}\\
\vspace*{0.5\baselineskip} 
{\large
%------------------------------------------------------------------------------------
    داده ساختار‌های پایه
%------------------------------------------------------------------------------------
}

\HRule \\[1.5cm]

\vfill

\begin{minipage}{0.4\textwidth}
\begin{center} \Large
%------------------------------------------------------------------------------------
    استاد:
%------------------------------------------------------------------------------------
\end{center}
\begin{center} \large
%------------------------------------------------------------------------------------
    دکتر روستایی
%------------------------------------------------------------------------------------
\end{center}
\end{minipage}

\vspace*{2\baselineskip} 

\begin{minipage}{0.4\textwidth}
\begin{center} \Large
%------------------------------------------------------------------------------------
    استادیار‌:
%------------------------------------------------------------------------------------
\end{center}
\begin{center} \large
%------------------------------------------------------------------------------------
عرشیا عموزاد\\
%------------------------------------------------------------------------------------
\end{center}
\end{minipage}

\vspace*{12\baselineskip} 

\today

\end{center}


\end{titlepage}

\begin{questions}

\question
یک ماتریس اسپارس، ماتریسی است که اکثر عناصر آن صفر هستند. به عبارت دیگر، تعداد عناصر غیرصفر آن در مقایسه با کل عناصر ماتریس بسیار کم است.
\newline
ماتریس معمولی (متراکم): همه یا بیشتر خانه‌ها پر از عدد هستند.
\newline
ماتریس اسپارس (خلوت): بیشتر خانهها خالی (صفر) هستند.
\newline
\vspace*{1\baselineskip}
\textbf{تبدیل ماتریس اسپارس به متراکم}
    \[    
    \begin{bmatrix}
        0 & 0 & 3 & 0 & 0 & 0 \\
        1 & 0 & 0 & 0 & 0 & 8 \\
        0 & 0 & 0 & 5 & 0 & 0 \\
        9 & 0 & 0 & 0 & 0 & 12\\
        0 & 0 & 0 & 10 & 0 & -5
    \end{bmatrix}
    \]

    \begin{center}
        \(\Downarrow\)
    \end{center}

    \[
    \begin{bmatrix}
        5 & 6 & 7 \\
        1 & 0 & 1 \\
        1 & 5 & 8 \\
        2 & 3 & 5 \\
        3 & 0 & 9 \\
        3 & 5 & 12 \\
        4 & 3 & 10 \\
        4 & 5 & -5


    \end{bmatrix}
    \]
    سطر اول ماتریس متراکم به ترتیب تعداد سطر، تعداد ستون و تعداد درایه های غیرصفر را نشان می‌دهد.\\
    سطر های دیگر نیز به ترتیب شماره سطر، شماره ستون و مقدار درایه های غیر صفر ماتریس اسپارس را نشان میدهند.


    \begin{enumerate}
        \item الگوریتمی برای محاسبه‌ی ترانهاده‌ی ماتریس اسپارس \(M_{p*q}\) بنویسید.
        \item متوسط تعداد جستجو های مورد نیاز برای جستجوی دودویی موفق در ماتریس زیر چقدر است؟
        \[
        \begin{bmatrix}
        0   & 0   & 0   & 120  & 0   & 0   & 0  \\
        0   & 0   & 0   & 0   & 0   & 0   & 101 \\
        0   & 0   & 0   & 0   & 83  & 0   & 0   \\
        0   & 0   & 0   & 0   & 0   & 54   & 0  \\
        0   & 0   & 0   & 0   & 0   & 0   & 0   \\
        0   & 9   & 0   & 0   & 0   & 0   & 0   \\
        0   & 0   & 0   & 0   & 0   & 7  & 0    \\
        0   & 0   & 0   & 0   & 3   & 0   & 0   \\
        0   & 0   & 2   & 0   & 0   & 0   & 0   \\
        1   & 0   & 0   & 0   & 0   & -6   & 0  \\
        \end{bmatrix}
    \]

    \end{enumerate}
\vspace*{1\baselineskip}

\question 
برای ضرب بهینه ماتریس های زیر، آن ها را پرانتزبندی کنید.\\
\(A_{10*2}*B_{2*25}*C_{25*3}*D_{3*4}\)



\vspace*{1\baselineskip}

\question
    عبارت میانوندی زیر را به پسوندی تبدیل کنید.
    \(A*(B-D)/E-F(G+F/K)\)
\vspace*{1\baselineskip}
% 40
\question
سه پشته‌ی \lr{\(S_1, S_2, S_3\)} هر یک حاوی دو عدد به شکل زیر موجود هستند.\\
\[
\begin{array}{|c|}
\\ \hline
1 \\ \hline
2 \\ \hline

\end{array}
\qquad
\begin{array}{|c|}
\\ \hline
3 \\ \hline
4 \\ \hline

\end{array}
\qquad
\begin{array}{|c|}
  \\ \hline
5 \\ \hline
6 \\ \hline
\end{array}
\]
دو عملگر \(poppush(i, j)\) و \(pop(i)\) بصورت زیر تعریف شده‌اند.
\(poppush(i, j)\) یک قلم از پشته \(S_i\) حذف و به پشته \(S_j\) اضافه می‌کند.
\(pop(i)\) یک قلم از پشته \(S_i\) حذف و سپس آن را چاپ می کند.
برای چاپ اعداد 1 تا 6 به صورت 1و 3و 5و 2و 4و 6 عملگر \(poppush\)
بایستی حداقل چند بار مورد استفاده قرار گیرد؟

\newpage

\question
تابع زیر چه کاری انجام می‌دهد؟ پیچیدگی زمانی آن را پیدا کنید.(ورودی تابع گره آغازین لیست پیوندی یکطرفه است)\\
\lr{\lstinputlisting[language=Python]{code/palindrome.py}}

\question
الگوریتمی ارائه دهید که بدون تغییر ساختار یک لیست پیوندی تشخیص دهد این لیست دور دارد یا خیر.
\vspace*{\baselineskip}

\question
عبارت \(\{x+(y-[a+b]*c-[(d+e)])\}/(j-(K-[L-n]))\) را در نظر بگیرید.
می‌خواهیم با استفاده از یک پشته بررسی کنیم که آیا پرانتز، کروشه و آکولادها بدرسی تطبیق می‌شوند یا نه. پشته مورد استفاده حداقل بایستی گنجایش چند عنصر را داشته باشد؟
\begin{checkboxes}
    \choice 4
    \choice 9
    \choice 18
    \choice 27
\end{checkboxes}

\end{questions}

\end{document}














