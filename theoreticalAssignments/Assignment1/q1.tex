
\documentclass[addpoints]{exam} % addpoints برای نمایش نمره


\RequirePackage[dvipsnames]{xcolor}
\RequirePackage{tcolorbox}
\RequirePackage{listings}
\usepackage{xepersian}
\usepackage{geometry}
\usepackage{graphicx}
\usepackage{enumitem}
\usepackage[utf8]{inputenc}
\usepackage{amsmath, amssymb}



\renewcommand{\labelitemi}{\scalebox{0.7}{$\bullet$}}

\geometry{
  a4paper,
  left=25mm,
  right=25mm,
  top=25mm,
  bottom=25mm
}

\newtcolorbox{mathbox}[1][]{
  colback=gray!5,
  left=1mm,
  right=1mm,
  leftrule=0pt,
  rightrule=1pt,
  toprule=0pt,
  bottomrule=0pt,
  sharp corners
}


\lstset{
    framerule=1pt,
    frame=tb,
    emphstyle={\small\ttfamily\bfseries\color{Orange}},
    numbers=left,
    numberstyle= \tiny\color{black},
    basicstyle = \small\ttfamily,
    keywordstyle    = \bfseries\color{BrickRed},
    identifierstyle = \bfseries\color{black},
    stringstyle     = \bfseries\color{ForestGreen},
    commentstyle    = \bfseries\color{Violet},
    breaklines      =   true,
    columns         =   fixed,
    basewidth       =   .5em,
    backgroundcolor=\color{Gray!5},
    tabsize=2,
    showspaces=false,
    showstringspaces=false,
}

\settextfont[Scale=1.1,Path=../../font/,Extension=.ttf,BoldFont=B Zar Bold]{B-NAZANIN}
\setlatintextfont[Scale=0.900]{Times New Roman}




\runningfooter{}{\thepage}{}
\firstpagefooter{}{\thepage}{}

\begin{document}


\begin{titlepage}
    \newcommand{\HRule}{\rule{\linewidth}{0.5mm}}

\newgeometry{top=1cm, bottom=2cm, left=1cm, right=1cm} 

\begin{flushright}
\begin{minipage}{0.15\textwidth}
\includegraphics[scale=.1]{../../figure/logo.png}
\end{minipage}
\begin{minipage}{0.5\textwidth}
\begin{flushright}
\textbf{\large انجمن علمی مهندسی کامپیوتر}\\
\large دانشگاه مازندران
\end{flushright}
\end{minipage}
\end{flushright}



\begin{center}

\vspace*{8\baselineskip}

{\LARGE 
%------------------------------------------------------------------------------------
  به نام خدا
%------------------------------------------------------------------------------------
}

\HRule \\[0.4cm]
\textbf{\Huge
%------------------------------------------------------------------------------------
    تمرین اول ساختمان داده‌ها و الگوریتم
%------------------------------------------------------------------------------------
}\\
\vspace*{0.5\baselineskip} 
{\large
%------------------------------------------------------------------------------------
    الگوریتم‌ها و حل مسئله - پیچیدگی زمانی - الگوریتم های بازگشتی
%------------------------------------------------------------------------------------
}

\HRule \\[1.5cm]

\vfill

\begin{minipage}{0.4\textwidth}
\begin{center} \Large
%------------------------------------------------------------------------------------
    استاد:
%------------------------------------------------------------------------------------
\end{center}
\begin{center} \large
%------------------------------------------------------------------------------------
    دکتر روستایی
%------------------------------------------------------------------------------------
\end{center}
\end{minipage}

\vspace*{2\baselineskip} 

\begin{minipage}{0.4\textwidth}
\begin{center} \Large
%------------------------------------------------------------------------------------
    استادیار‌:
%------------------------------------------------------------------------------------
\end{center}
\begin{center} \large
%------------------------------------------------------------------------------------
عرشیا عموزاد\\
%------------------------------------------------------------------------------------
\end{center}
\end{minipage}

\vspace*{12\baselineskip} 

\today

\end{center}


\end{titlepage}

\begin{questions}


\question
توابع زیر را از نظر رشد با هم مقایسه کنید.\\
\begin{flushleft}
    \(g(n) = n^{\log n}\)\\
    \(f(n) = (log n)^n (n)\)
\end{flushleft}
\vspace*{1\baselineskip}




\question 
کدام یک از گزاره های زیر درباره‌ی توابع ذکر شده صحیح است؟\\
    \begin{flushleft}
       \(f(n)=4^{\log n}\)\\ \(g(n)=(\log n)^{\log n}\)\\ \(h(n) = \log^2 n\)
    \end{flushleft}

\begin{checkboxes} 
    \choice \(f(n) = O(g(n), g(n) = \Omega(h(n)))\)
    \choice \(g(n) = \Omega(h(n), h(n) = \Omega(f(n)))\)
    \choice \(f(n) = O(h(n), g(n) = \Omega(f(n)))\)
    \choice \(h(n) = O(g(n), f(n) = \Theta(g(n)))\)
\end{checkboxes}
\vspace*{1\baselineskip}

% \question
% کد زیر الگوریتم جستجوی خطی را نشان می دهد\\
% فرض کنید با احتمال 50 درصد عنصر \lr{target} در آرایه‌ی \lr{arr} است. بطور متوسط چند عنصر برای یافتن \lr{target} بررسی می‌شوند؟\\
% \lr{\lstinputlisting[language=Python]{code/linearSearch.py}}


% \question
% آیا می‌توان ادعا کرد که برای هر دو تابع \(f(n)\) و \(g(n)\) یا \(f(n) = O(g(n))\) یا \(g(n) = O(f(n))\) یا هر دو؟\\


\question
توابع زیر را از نظر نرخ رشد به ترتیب صعودی مرتب کنید.\\
\begin{flushleft}
    \(f_1(n) = n^{1.01}\), \(f_2(n) = 2^{\sqrt{n}}\), \(f_3(n) = (\log{n})^{\log{n}}\), \(f_4(n) = n^{\log{n}}\), \(f_5(n) = n^{\log{\log{n}}}\)
    
\end{flushleft}
\vspace*{1\baselineskip}

\question
در یک زمستان سرد، خرس قطبی \lr{n} قطعه گوشت دقیقا به اندازه‌های 1، 2 تا \lr{n} را در غاری ذخیره کرده است.
او هر روز یکی از این قطعه گوشت ها را به صورت تصادفی انتخاب می‌کند. اگر اندازه‌ی گوشت عدد فردی بود، آن را کاملا می‌خورد.
 اگر زوج بود، آن را دقیقا نصف می‌کند، یک نصف آن را می‌خورد و نصف دیگر را مجددا در غار قرار می‌دهد.
  اگر گوشتی موجود نباشد خرس می‌میرد. با این الگوریتم، برای 
\lr{n} های خیلی بزرگ روزهای باقیمانده از عمر خرس قطبی تابع کدام یک از گزینه های خواهد بود؟\\

\begin{oneparcheckboxes}
    \choice \(O(n)\)
    \choice \(O(\log n)\)
    \choice \(O(n \log n)\)
    \choice \(O(n^2)\)

\end{oneparcheckboxes}
\vspace*{1\baselineskip}

\question
پیچیدگی زمانی توابع زیر را پیدا کنید.

\begin{enumerate}[leftmargin=*]
    \item \(T(n) = 4T(\frac{n}{2}) + \log^2{n!}\)  % صفحه 35 پارت وسط
    % \item \(T(n) = 3\sqrt[3]{n^2}T(\sqrt[3]{n})+n.\log n\) % ابتدای صفحه 36
    % \item \(T(n) = T(\frac{n}{3}) + T(\frac{n}{6}) + n^{\sqrt{\log n}}\) % f page 44
    % % \item \(H(n) = \frac{1}{2-H(n-1)}\), \(H(0) = 0\) % h page 44
    \item \(f(n) = \log(n!)\)
\end{enumerate}

\newpage


% \question
% الگوریتم زیر \(\binom{n}{m}\) را برای اعداد صحیح \(n \geq m\) محاسبه میکند: \\
% \lr{\lstinputlisting[language=Python]{code/selection.py}}
% برای \(n > m\) دلخواه عمل جمع چند بار تکرار می‌شود؟   \\
% % سوال 5 صفحه 47
% \begin{oneparcheckboxes}
%     \choice \(\binom{n}{m}\)
%     \choice \(nm\)
%     \choice \(\binom{n}{m} -1\)
%     \choice \(n(n-m)\)

% \end{oneparcheckboxes}

\question
درباره‌ی اجرای الگوریتم مرتب سازی ادغامی \lr{(Merge Sort)} بر روی آرایه‌ای به طول \lr{n} به سوالات زیر پاسخ دهید. 
\begin{enumerate}
    \item اگر در مرحله‌ی تقسیم، الگوریتم به جای نصف شدن دقیق به نسبت های دلخواه مانند \(\frac{1}{3}\) و \(\frac{2}{3}\)
     یا \(\frac{1}{4}\) و \(\frac{3}{4}\) تقسیم شود پیچیدگی زمانی الگوریتم چگونه تغییر می‌کند؟
    \item فرض کنید همه‌ی عناصر آرایه از قبل بصورت صعودی مرتب شده باشند. در این صورت تعداد مقایسه های الگوریتم نسبت به حالت تصادفی چه تغییری می‌کند؟
\end{enumerate}
\vspace*{1\baselineskip}

\question
با اجرای برنامه‌ی زیر چه عددی چاپ می‌شود؟ \\
\lr{\lstinputlisting[language=c]{code/ackerman.c}}


\end{questions}

\end{document}














